%%%%%%%%%%%%%%%%%%%%%%%%%%%%%%%%%%%%%%%%%%%%%%%%%%%%%%
% A template for Wiley article submissions.
% Developed by Overleaf. 
%
% Please note that whilst this template provides a 
% preview of the typeset manuscript for submission, it 
% will not necessarily be the final publication layout.
%
% Usage notes:
% The "blind" option will make anonymous all author, affiliation, correspondence and funding information.
% Use "num-refs" option for numerical citation and references style.
% Use "alpha-refs" option for author-year citation and references style.

\documentclass[num-refs]{wiley-article}

% added for highlighting
\usepackage{soul}

% added for margin notes
\usepackage{marginnote}

\bibliographystyle{vancouver-authoryear}

% Add additional packages here if required
\usepackage{siunitx}

\usepackage{float}

% added by MW
\graphicspath{{./figures/}}

\renewcommand{\thefigure}{S\arabic{figure}}

\begin{document}
\section*{SUPPORTING INFORMATION}

\begin{figure}[H]
\begin{center}
\includegraphics[width=0.7\textwidth]{figure_s1.eps}
\caption{8 simulated spectra (SNR=25) with random phase and linearly increasing frequency shifts up to 10Hz. These spectra are typical of those used for the results presented in Figure 2.}
\label{freq}
\end{center}
\end{figure}

\begin{figure}[h]
\begin{center}
\includegraphics[width=0.95\textwidth]{figure_s2.eps}
\caption{a) 32 overlaid simulated spectra (SNR=15) with unstable baseline, random phase and linearly increasing frequency shifts up to 10Hz. b) TDSR corrected spectra and c) RATS corrected spectra. These spectra have a reduced baseline distortion strength of 40\% relative to Figure 4.}
\label{freq}
\end{center}
\end{figure}
\end{document}

