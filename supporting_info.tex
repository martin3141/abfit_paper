%%%%%%%%%%%%%%%%%%%%%%%%%%%%%%%%%%%%%%%%%%%%%%%%%%%%%%
% A template for Wiley article submissions.
% Developed by Overleaf. 
%
% Please note that whilst this template provides a 
% preview of the typeset manuscript for submission, it 
% will not necessarily be the final publication layout.
%
% Usage notes:
% The "blind" option will make anonymous all author, affiliation, correspondence and funding information.
% Use "num-refs" option for numerical citation and references style.
% Use "alpha-refs" option for author-year citation and references style.

\documentclass[num-refs]{wiley-article}

% added for highlighting
\usepackage{soul}

% added for margin notes
\usepackage{marginnote}

\bibliographystyle{vancouver-authoryear}

% Add additional packages here if required
\usepackage{siunitx}

\usepackage{float}

% added by MW
\graphicspath{{./figures/}}

\renewcommand{\thefigure}{S\arabic{figure}}
\renewcommand{\thetable}{S\arabic{table}}

\begin{document}
\section*{SUPPORTING INFORMATION}

\begin{figure}[H]
  \begin{center}
    \includegraphics[width=1\textwidth]{figS1}
    \caption{TODO}
  \end{center}
\end{figure}

\begin{figure}
  \begin{center}
    \includegraphics[width=1\textwidth]{figS2}
    \caption{TODO}
  \end{center}
\end{figure}

\begin{figure}
  \begin{center}
    \includegraphics[width=1\textwidth]{figS3}
    \caption{TODO}
  \end{center}
\end{figure}

\begin{figure}
  \begin{center}
    \includegraphics[width=1\textwidth]{figS4}
    \caption{TODO}
  \end{center}
\end{figure}

\begin{figure}
  \begin{center}
    \includegraphics[width=1\textwidth]{figS5}
    \caption{TODO}
  \end{center}
\end{figure}

\begin{figure}
  \begin{center}
    \includegraphics[width=1\textwidth]{figS6}
    \caption{TODO}
  \end{center}
\end{figure}

\begin{figure}
  \begin{center}
    \includegraphics[width=1\textwidth]{figS7}
    \caption{ABfit results from a 2D MRSI semi-LASER acquisition. A) Orthogonal T1 MRI slices intersecting the analysis region --- shown as a colored tNAA / tCr metabolite map overlay. Linear regression of key metabolite ratios with the gray matter fraction are shown in parts B, C and D. The line of best fit is plotted in blue with the 95\% confidence region in gray.}
    \label{mrsi_res}
  \end{center}
\end{figure}

\begin{figure}
  \begin{center}
    \includegraphics[width=1\textwidth]{figS8}
    \caption{ABfit result plot for a voxel within a 2D MRSI semi-LASER acquisition with an asymmetric lineshape. A) default analysis with an asymmetric lineshape fit model, B) analysis with the lineshape asymmetry parameter ($a_{g}$) restricted to $\pm 0.0001$.}
    \label{lineshape_res}
  \end{center}
\end{figure}

\begin{table}[ht]
\begin{center}
\begin{tabular}{l r}
  \hline
  Metabolite & Amplitude (mM) \\
  \hline
  Alanine & 0.80 \\
  Aspartate & 1.00 \\
  Creatine & 7.50 \\
  gamma-Aminobutyric acid & 1.50 \\
  Glucose & 1.50 \\
  Glutamine & 4.50 \\
  Glutamate & 9.25 \\
  Glutathione & 2.25 \\
  Glycerophosphorylcholine & 1.00 \\
  Lactate & 0.60 \\
  myo-Inositol & 6.50 \\
  N-acetylaspartate & 12.25 \\
  N-acetylaspartylglutamate & 1.50 \\
  Phosphocholine & 0.60 \\
  Phosphocreatine & 4.25 \\
  scyllo-Inositol & 0.35 \\
  Taurine & 4.00 \\
  \hline
\end{tabular}
\end{center}
\caption{Metabolite concentrations consistent with levels measured in normal brain tissue.}
\label{metab_tab}
\end{table}

\begin{table}[ht]
\begin{center}
\begin{tabular}{l r r r}
  \hline
  Signal & Frequency (PPM) & FWHM (PPM) & Amplitude (a.u.) \\
  \hline
  Lip13a & 1.28 & 0.15 & 2.0 \\
  Lip13b & 1.28 & 0.089 & 2.0 \\
  Lip09 & 0.89 & 0.14 & 3.0 \\
  MM09 & 0.91 & 0.14 & 3.0 \\
  Lip20 & 2.04 & 0.15 & 1.33 \\
  Lip20 & 2.25 & 0.15 & 0.67 \\
  Lip20 & 2.80 & 0.20 & 0.87 \\
  MM20 & 2.08 & 0.15 & 1.33 \\
  MM20 & 2.25 & 0.20 & 0.33 \\
  MM20 & 1.95 & 0.15 & 0.33 \\
  MM20 & 3.00 & 0.20 & 0.4 \\
  MM12 & 1.21 & 0.15 & 2.0 \\
  MM14 & 1.43 & 0.17 & 2.0 \\
  MM17 & 1.67 & 0.15 & 2.0 \\
  \hline
\end{tabular}
\end{center}
\caption{Parameters used to generate the individual simulated lipid and macromolecule basis signals. Listed components with the same name were summed to form a composite signal.}
\label{lip_mm_tab}
\end{table}

\end{document}
